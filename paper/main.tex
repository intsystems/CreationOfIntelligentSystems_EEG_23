\documentclass[a4paper, 12pt]{article}

\usepackage{arxiv}

\usepackage[T2A]{fontenc}
\usepackage[utf8]{inputenc}
\usepackage[english, russian]{babel}
% \usepackage{cmap}
\usepackage{url}
\usepackage{booktabs}
\usepackage{nicefrac}
\usepackage{microtype}
\usepackage{lipsum}
\usepackage{graphicx}
\usepackage{subfig}
\usepackage[square,sort,comma,numbers]{natbib}
\usepackage{doi}
\usepackage{multicol}
\usepackage{multirow}
\usepackage{tabularx}

\usepackage{tikz}
\usetikzlibrary{matrix}

% Algorithms
\usepackage{algpseudocode}
\usepackage{algorithm}

%% Шрифты
\usepackage{euscript} % Шрифт Евклид
\usepackage{mathrsfs} % Красивый матшрифт
\usepackage{extsizes} % Возможность сделать 14-й шрифт

\usepackage{makecell} % diaghead in a table
\usepackage{amsmath,amsfonts,amssymb,amsthm,mathtools,dsfont}
\usepackage{icomma}

\newcommand{\bz}{\mathbf{z}}
\newcommand{\bff}{\mathbf{f}}
\newcommand{\bx}{\mathbf{x}}
\newcommand{\by}{\mathbf{y}}
\newcommand{\bv}{\mathbf{v}}
\newcommand{\bw}{\mathbf{w}}
\newcommand{\ba}{\mathbf{a}}
\newcommand{\bb}{\mathbf{b}}
\newcommand{\bp}{\mathbf{p}}
\newcommand{\bq}{\mathbf{q}}
\newcommand{\bt}{\mathbf{t}}
\newcommand{\bg}{\mathbf{g}}
\newcommand{\bu}{\mathbf{u}}
\newcommand{\bT}{\mathbf{T}}
\newcommand{\bX}{\mathbf{X}}
\newcommand{\bZ}{\mathbf{Z}}
\newcommand{\bS}{\mathbf{S}}
\newcommand{\bH}{\mathbf{H}}
\newcommand{\bW}{\mathbf{W}}
\newcommand{\bM}{\mathbf{M}}
\newcommand{\bY}{\mathbf{Y}}
\newcommand{\bU}{\mathbf{U}}
\newcommand{\bQ}{\mathbf{Q}}
\newcommand{\bP}{\mathbf{P}}
\newcommand{\bA}{\mathbf{A}}
\newcommand{\bB}{\mathbf{B}}
\newcommand{\bC}{\mathbf{C}}
\newcommand{\bE}{\mathbf{E}}
\newcommand{\bF}{\mathbf{F}}
\newcommand{\bomega}{\boldsymbol{\omega}}
\newcommand{\btheta}{\boldsymbol{\theta}}
\newcommand{\bgamma}{\boldsymbol{\gamma}}
\newcommand{\bdelta}{\boldsymbol{\delta}}
\newcommand{\bPsi}{\boldsymbol{\Psi}}
\newcommand{\bpsi}{\boldsymbol{\psi}}
\newcommand{\bxi}{\boldsymbol{\xi}}
\newcommand{\bchi}{\boldsymbol{\chi}}
\newcommand{\bzeta}{\boldsymbol{\zeta}}
\newcommand{\blambda}{\boldsymbol{\lambda}}
\newcommand{\beps}{\boldsymbol{\varepsilon}}
\newcommand{\bZeta}{\boldsymbol{Z}}
% mathcal
\newcommand{\cX}{\mathcal{X}}
\newcommand{\cY}{\mathcal{Y}}
\newcommand{\cW}{\mathcal{W}}

\newcommand{\dH}{\mathds{H}}
\newcommand{\dR}{\mathds{R}}
% transpose
\newcommand{\T}{^{\mathsf{T}}}

% \renewcommand{\shorttitle}{\textit{arXiv} Шаблон}
\renewcommand{\epsilon}{\ensuremath{\varepsilon}}
\renewcommand{\phi}{\ensuremath{\varphi}}
\renewcommand{\kappa}{\ensuremath{\varkappa}}
\renewcommand{\le}{\ensuremath{\leqslant}}
\renewcommand{\leq}{\ensuremath{\leqslant}}
\renewcommand{\ge}{\ensuremath{\geqslant}}
\renewcommand{\geq}{\ensuremath{\geqslant}}
\renewcommand{\emptyset}{\varnothing}

\usepackage{hyperref}
% \usepackage[usenames,dvipsnames,svgnames,table,rgb]{xcolor}

\hypersetup{
	unicode=true,
	pdftitle={A template for the arxiv style},
	pdfsubject={q-bio.NC, q-bio.QM},
	pdfauthor={David S.~Hippocampus, Elias D.~Striatum},
	pdfkeywords={First keyword, Second keyword, More},
	colorlinks=true,
	linkcolor=black,        % внутренние ссылки
	citecolor=blue,         % на библиографию
	filecolor=magenta,      % на файлы
	urlcolor=blue           % на URL
}

\graphicspath{{../figures/}}

\usepackage{enumitem} % Для модификаций перечневых окружений

\theoremstyle{definition} % "Определение"
\newtheorem{definition}{Опр.}[section]

\usepackage{etoolbox}

\makeatletter
\expandafter\patchcmd\csname\string\algorithmic\endcsname{\itemsep\z@}{\itemsep=1.5mm}{}{}
\makeatother
\renewcommand{\abstractname}{Аннотация}

\title{Генеративные модели декодирования временных рядов}

\author{Владимиров Эдуард \\
	\texttt{vladimirov.ea@phystech.edu} \\	
	\And
	бартенев Павел \\
	\texttt{example@phystech.edu} \\
	\And
	Чумаченко Арина \\
	\texttt{example@phystech.edu}
}
\date{\today}

\begin{document}
\maketitle

\begin{abstract}
	Это исследование нацелено на продвижение области генеративных моделей декодирования временных рядов и представляет новый подход к улучшению процессов принятия решений в сложных системах. Основная цель состоит в двух аспектах: в принятии информированных классификационных решений и, при необходимости, в их эффективном отклонении на основе критерия отклонения от решения, выведенного из несоответствий в наблюдениях внутри созданных сценариев. Наша работа основана на ключевых предположениях, включая анализ относительно коротких временных рядов с значительными изменчивостями, системными ошибками и сложными взаимосвязями. Кроме того, мы рассматриваем данные, происходящие из разных источников, включая экзогенные факторы, управляемые сигналы, решения и поведенческие паттерны, все это внутри структурированного времени. Мы демонстрируем универсальность нашего подхода через применение в различных областях, таких как анализ спортивных игр, интерфейсы мозг-компьютер и управление рисками. Объединяя эти принципы с нашими генеративными моделями, мы стремимся повысить качество классификации поведения и процессов принятия решений в различных областях, способствуя улучшению понимания и прогнозирования динамических систем.
\end{abstract}


\keywords{нейронное кодирование \and нейронное декодирование \and D4 \and SSM}

%%%%%%%%%%%%%%%%%%%%%%%%%%%%%%%%%%%%%%%%%%%%%%%%%%%%%%%%%%%%%%%%%%%%%%%%%%%%%%%%%%%%%%%%%%%%%%%%%%%%%%%%%%%%%%%%%%%%%%%%%%%%%%%%%%
\section{Введение}
	Появление генеративных моделей декодирования временных рядов открывает новую эру усовершенствования процессов принятия решений в различных областях. Это исследование начинает свой путь, чтобы внести свой вклад в это развивающееся направление, черпая вдохновение из недавних достижений в области анализа временных рядов и генеративных моделей.
	
	Несколько важных работ подготовили почву для данного исследования. В работе "Прямые дискриминативные декодерные модели для анализа высокоразмерных динамических нейральных данных" (Резаи и др., 2022) авторы глубоко погружаются в сложности декодирования высокоразмерных нейральных данных, заложив основы для инновационных применений в области интерфейсов мозг-компьютер (ИМК). Кроме того, "Параметрические регрессоры гауссовских процессов" (Янковяк и др., 2020) представляют техники, которые можно интегрировать в нашу генеративную модель для моделирования временных рядов, проявляющих различные изменчивости и систематические ошибки.
	
	Наша работа выходит за рамки теоретических исследований, нацелена на решение практических задач в сложных системах. Мы представляем критерий отклонения от решения, основанный на несоответствиях в наблюдениях внутри созданных сценариев, концепцию, аналогичную принципу классификации "один класс" из работы "Глубокие прямые дискриминативные декодеры для анализа высокоразмерных временных рядов" (Резаи, 2023). Этот критерий дарит нашим генеративным моделям не только способность принимать информированные классификационные решения, но и эффективно отклонять решения, когда сталкиваются с неожиданными паттернами или выбросами.
	
	Кроме того, мы учитываем ряд предположений и сценариев, включая короткие временные ряды, существенные корреляции между временными рядами и структурированные временные линии. Эти предположения соответствуют принципам, рассмотренным в работе "Интуитивное руководство по регрессии гауссовских процессов" (Уанг, 2021), где освещаются основные понятия гауссовских процессов - важного компонента нашего подхода к моделированию.
	
	Потенциал применения нашего исследования охватывает широкий спектр областей, от анализа спортивных игр до ИМК и управления рисками на финансовых рынках. Комбинируя принципы из этих влиятельных работ с нашими генеративными моделями, мы стремимся внести вклад в усовершенствование процессов принятия решений, классификации поведения и понимание динамических систем в различных областях.
	
	Основная идея: создать модель автокодировщик для генерации сигналов ЭЭГ: кодировщик - модель D4, декодер - модель из семейства SSM, является частью decision-making процесса.

%%%%%%%%%%%%%%%%%%%%%%%%%%%%%%%%%%%%%%%%%%%%%%%%%%%%%%%%%%%%%%%%%%%%%%%%%%%%%%%%%%%%%%%%%%%%%%%%%%%%%%%%%%%%%%%%%%%%%%%%%%%%%%%%
\section{Обозначения}

TODO
%	\begin{itemize}
%	\end{itemize}

%%%%%%%%%%%%%%%%%%%%%%%%%%%%%%%%%%%%%%%%%%%%%%%%%%%%%%%%%%%%%%%%%%%%%%%%%%%%%%%%%%%%%%%%%%%%%%%%%%%%%%%%%%%%%%%%%%%%%%%%%%%%%%%
\section{Постановка задачи классификации сигнала ЭКоГ}
TODO

%%%%%%%%%%%%%%%%%%%%%%%%%%%%%%%%%%%%%%%%%%%%%%%%%%%%%%%%%%%%%%%%%%%%%%%%%%%%%%%%%%%%%%%%%%%%%%%%%%%%%%%%%%%%%%%%%%%%%%%%%%%%%%%%%%%%%%%
\section{Обзор литературы}
TODO

%%%%%%%%%%%%%%%%%%%%%%%%%%%%%%%%%%%%%%%%%%%%%%%%%%%%%%%%%%%%%%%%%%%%%%%%%%%%%%%%%%%%%%%%%%%%%%%%%%%%%%%%%%%%%%%%%%%%%%%%%%%%%%%%%%%%%%%
\section{Модели пространства состояний}

%%%%%%%%%%%%%%%%%%%%%%%%%%%%%%%%%%%%%%%%%%%%%%%%%%%%%%%%%%%%%%%%%%%%%%%%%%%%%%%%%%%%%%%%%%%%%%%%%%%%%%%%%%%%%%%%%%%%%%%%%%%%%%%%
\section{Вычислительный эксперимент}
Целью эксперимента является TODO.

\subsection{Экспериментальные данные}
TODO

\subsection{Условия проведения эксперимента}
TODO

\subsection{Анализ ошибки}
TODO


%%%%%%%%%%%%%%%%%%%%%%%%%%%%%%%%%%%%%%%%%%%%%%%%%%%%%%%%%%%%%%%%%%%%%%%%%%%%%%%%%%%%%%%%%%%%%%%%%%%%%%%%%%%%%%%%%%%%%%%%%%%%%%%%%%%%%%%
\section{Заключение}
TODO.

%%%%%%%%%%%%%%%%%%%%%%%%%%%%%%%%%%%%%%%%%%%%%%%%%%%%%%%%%%%%%%%%%%%%%%%%%
\addcontentsline{toc}{section}{\protect\numberline{}Список литературы}
\bibliographystyle{unsrtnat}
\bibliography{references.bib}

\end{document} 